\documentclass[a4paper]{article}
\usepackage[T2A]{fontenc}
\usepackage[utf8]{inputenc}
\usepackage[english,russian]{babel}
\usepackage[pdf]{graphviz}
\usepackage[table]{xcolor}
\usepackage{morewrites}
\usepackage{verbatim}
\usepackage{amsmath}
\usepackage{calc}
\usepackage{tikz}
\usepackage{pgfplots}
\usepackage{import}
\usepackage{xifthen}
\usepackage{pdfpages}
\usepackage{transparent}

\usepackage{xpatch}
\makeatletter
\newcommand*{\addFileDependency}[1]{% argument=file name and extension
  \typeout{(#1)}
  \@addtofilelist{#1}
  \IfFileExists{#1}{}{\typeout{No file #1.}}
}
\makeatother
\xpretocmd{\digraph}{\addFileDependency{#2.dot}}{}{}



\author{Чуворкин Михаил А-13а-19}
\title{Домашняя работа по дисциплине Теоретические модели вычслений №1}


\begin{document}
\nocite{*}
\maketitle

\section{Задание №1. Построить конечный автомат, распознающий язык.}

\subsection{Язык 1.}
$$ L = \left\{ w \in \left\{ a, b, c \right\}^{*} \mid \left| w \right|_{c} = 1 \right\} $$

Данный язык включает все слова из букв $\left\{ a, b, c \right\}$, но содержащие только одну букву $c$. 

\digraph[scale=0.5]{esm11}{
	rankdir=LR; 
	node [shape=none]; start;
	node [shape=circle]; 1;
	node [shape=doublecircle]; 2;
	start -> 1;
	1 -> 2 [label = "c"];
	1 -> 1 [label = "a,b"];
	2 -> 2 [label = "a,b"];
}






\subsection{Язык 2.}
$$ L = \left\{ w \in \left\{ a, b \right\}^{*} \mid \left| w \right|_{a} \leq 2, \left| w \right|_{b} \geq 2 \right\} $$

Предствим, что есть решетка. При поглощении $a$ происходит перемещение вниз по решетке, а при поглощении $b$ переход вправо по решетке.

\digraph[scale=0.5]{esm12}{
	rankdir=LR; 
	node [shape=none]; start;
	node [shape=circle]; 1 2 4 5 7 8;
	node [shape=doublecircle]; 3 6 9;
	start -> 1;
	1 -> 2 [label = "b"];
	1 -> 4 [label = "a"];
	2 -> 3 [label = "b"];
	2 -> 5 [label = "a"];
	3 -> 3 [label = "b"];
	3 -> 6 [label = "a"];

	4 -> 5 [label = "b"];
	4 -> 7 [label = "a"];
	5 -> 6 [label = "b"];
	5 -> 8 [label = "a"];
	6 -> 6 [label = "b"];
	6 -> 9 [label = "a"];

	7 -> 8 [label = "b"];
	8 -> 9 [label = "b"];
	9 -> 9 [label = "b"];
}





\subsection{Язык 3.}
$$ L = \left\{ w \in \left\{ a, b \right\}^{*} \mid \left| w \right|_{a} \neq \left| w \right|_{b} \right\} $$

Нельзя построить автомат, так как для распознавания этого языка требуется запоминать количество символов.


\subsection{Язык 4.}
$$ L = \left\{ w \in \left\{ a, b \right\}^{*} \mid ww=www \right\} $$

В данном случае язык будет состоять только из пустого символа, так как условая спарведливы только для $w = \varepsilon$


\digraph[scale=0.5]{esm13}{
	rankdir=LR; 
	node [shape=none]; start;
	node [shape=doublecircle]; 1;
	start -> 1;
}

\section{Задание №2. Построить конечный автомат, используя прямое произведение.}

\subsection{Язык 1.}
$$ L_{1} = \left\{ w \in \left\{ a, b \right\}^{*} \mid \left| w \right|_{a} \geq 2 \wedge \left| w \right|_{b} \geq 2 \right\} $$

Рассмотрим языки, описывающие части условия языка $L$:

$ A_{1} = \left\{ w \in \left\{ a, b \right\}^{*} \mid \left| w \right|_{a} \geq 2 \right\} $

$ \Sigma_{A} = \left\{ a, b \right\}$
$ Q_{A} = \left\{ 1, 2, 3 \right\} $
$ s_{A} = \left\{ 1 \right\} $
$ T_{A} = \left\{ 3 \right\}$

\digraph[scale=0.5]{esm21A1}{
	rankdir=LR; 
	node [shape=none]; start;
	node [shape=circle]; 1 2;
	node [shape=doublecircle]; 3;
	start -> 1;
	1 -> 2 [label = "a"];
	1 -> 1 [label = "b"];
	2 -> 3 [label = "a"];
	2 -> 2 [label = "b"];
	3 -> 3 [label = "a,b"];
}


$ B_{1} = \left\{ w \in \left\{ a, b \right\}^{*} \mid \left| w \right|_{b} \geq 2 \right\} $

$ \Sigma_{B} = \left\{ a, b \right\}$
$ Q_{B} = \left\{ 1, 2, 3 \right\} $
$ s_{B} = \left\{ 1 \right\} $
$ T_{B} = \left\{ 3 \right\}$


\digraph[scale=0.5]{esm21B1}{
	rankdir=LR; 
	node [shape=none]; start;
	node [shape=circle]; 1 2;
	node [shape=doublecircle]; 3;
	start -> 1;
	1 -> 2 [label = "b"];
	1 -> 1 [label = "a"];
	2 -> 3 [label = "b"];
	2 -> 2 [label = "a"];
	3 -> 3 [label = "a,b"];
}

Тогда: $L_{1} = A_{1} \times B_{1}$

$ \Sigma = \Sigma_{A} \cup \Sigma_{B} = \left\{ a, b \right\}$

$ Q = Q_{A} \times Q_{B} = \left\{ 11, 12, 13, 21, 22, 23, 31, 32, 33 \right\}$

$ s = <s_{1}, s_{2}> = \left\{ 11 \right\}$

$ T = T_{A} \times T_{B} = \left\{ 33 \right\}$


Таблица состояний:
\begin{table}[!htbp]
\rowcolors{2}{gray!10}{gray!40}
\centering
\begin{tabular}{|l|l|l|l|}
\hline
$A$ & $B$ & $a$ & $b$ \\ \hline
1   & 1   & 21  & 12  \\ \hline
1   & 2   & 22  & 13  \\ \hline
1   & 3   & 23  & 13  \\ \hline
2   & 1   & 31  & 22  \\ \hline
2   & 2   & 32  & 23  \\ \hline
2   & 3   & 33  & 23  \\ \hline
3   & 1   & 31  & 32  \\ \hline
3   & 2   & 32  & 33  \\ \hline
3   & 3   & 33  & 33  \\ \hline
\end{tabular}
\caption{Таблица состояний (Задание 2.1)}
\end{table}


\digraph[scale=0.5]{esm21}{
	rankdir=LR; 
	node [shape=none]; start;
	node [shape=circle]; 11 12 13 21 22 23 31 32;
	node [shape=doublecircle]; 33;
	start -> 11;
	11 -> 12 [label = "b"];
	11 -> 21 [label = "a"];
	12 -> 13 [label = "b"];
	12 -> 22 [label = "a"];
	13 -> 13 [label = "b"];
	13 -> 23 [label = "a"];

	21 -> 22 [label = "b"];
	21 -> 31 [label = "a"];
	22 -> 23 [label = "b"];
	22 -> 32 [label = "a"];
	23 -> 23 [label = "b"];
	23 -> 33 [label = "a"];

	31 -> 32 [label = "b"];
	31 -> 31 [label = "a"];
	32 -> 33 [label = "b"];
	32 -> 32 [label = "a"];
	33 -> 33 [label = "b"];
}

\subsection{Язык 2.}
$$ L_{2} = \left\{ w \in \left\{ a, b \right\}^{*} \mid \left| w \right| \geq 3 \wedge \left| w \right| \text{нечетное} \right\} $$

Рассмотрим языки, описывающие части условия языка $L$:

$ A_{2} = \left\{ w \in \left\{ a, b \right\}^{*} \mid \left| w \right| \geq 3 \right\} $

$ \Sigma_{A} = \left\{ a, b \right\}$
$ Q_{A} = \left\{ 1, 2, 3, 4 \right\} $
$ s_{A} = \left\{ 1 \right\} $
$ T_{A} = \left\{ 4 \right\}$

\digraph[scale=0.5]{esm22A2}{
	rankdir=LR; 
	node [shape=none]; start;
	node [shape=circle]; 1;
	node [shape=circle]; 2;
	node [shape=circle]; 3;
	node [shape=doublecircle]; 4;
	start -> 1;
	1 -> 2 [label = "a,b"];
	2 -> 3 [label = "a,b"];
	3 -> 4 [label = "a,b"];
	4 -> 4 [label = "a,b"];
}


$ B_{2} = \left\{ w \in \left\{ a, b \right\}^{*} \mid  \left| w \right| \text{нечетное} \right\} $

$ \Sigma_{B} = \left\{ a, b \right\}$
$ Q_{B} = \left\{ 1, 2 \right\} $
$ s_{B} = \left\{ 1 \right\} $
$ T_{B} = \left\{ 2 \right\}$

\digraph[scale=0.5]{esm22B2}{
	rankdir=LR; 
	node [shape=none]; start;
	node [shape=circle]; 1;
	node [shape=doublecircle]; 2;
	start -> 1;
	1 -> 2 [label = "a,b"];
	2 -> 1 [label = "a,b"];
}

Тогда: $L_{2} = A_{2} \times B_{2}$

$ \Sigma = \left\{ a, b \right\}$

$ Q = \left\{ 11, 12, 21, 22, 31, 32, 41, 42 \right\}$

$ s = \left\{ 11 \right\}$

$ T = \left\{ 42 \right\}$


\begin{table}[!htbp]
\rowcolors{2}{gray!10}{gray!40}
\centering
\begin{tabular}{|l|l|l|l|}
\hline
$A$ & $B$ & $a$ & $b$ \\ \hline
1   & 1   & 22  & 22  \\ \hline
1   & 2   & 21  & 21  \\ \hline
2   & 1   & 32  & 32  \\ \hline
2   & 2   & 31  & 31  \\ \hline
3   & 1   & 42  & 42  \\ \hline
3   & 2   & 41  & 41  \\ \hline
4   & 1   & 42  & 42  \\ \hline
4   & 2   & 41  & 41  \\ \hline
\end{tabular}
\caption{Таблица состояний (Задание 2.2)}
\end{table}


\digraph[scale=0.5]{esm22}{
	rankdir=LR; 
	node [shape=none]; start;
	node [shape=circle]; 11 12 21 22 31 32 41;
	node [shape=doublecircle]; 42;
	start -> 11;
	11 -> 22 [label = "a,b"];
	12 -> 21 [label = "a,b"];
	21 -> 32 [label = "a,b"];
	22 -> 31 [label = "a,b"];
	31 -> 42 [label = "a,b"];
	32 -> 41 [label = "a,b"];
	41 -> 42 [label = "a,b"];
	42 -> 41 [label = "a,b"];

}


\subsection{Язык 3.}
$$ L_{3} = \left\{ w \in \left\{ a, b \right\}^{*} \mid \left| w \right|_{a} \text{четное} \wedge \left| w \right|_{b} \text{кратно трем} \right\} $$

Рассмотрим языки, описывающие части условия языка $L$:

$ A_{3} = \left\{ w \in \left\{ a, b \right\}^{*} \mid \left| w \right|_{a} \text{четное} \right\} $

$ \Sigma_{A} = \left\{ a, b \right\}$
$ Q_{A} = \left\{ 1, 2 \right\} $
$ s_{A} = \left\{ 1 \right\} $
$ T_{A} = \left\{ 1 \right\}$

\digraph[scale=0.5]{esm23A3}{
	rankdir=LR; 
	node [shape=none]; start;
	node [shape=circle]; 2;
	node [shape=doublecircle]; 1;
	start -> 1;
	1 -> 1 [label = "b"];
	1 -> 2 [label = "a"];
	2 -> 2 [label = "b"];
	2 -> 1 [label = "a"];
}


$ B_{3} = \left\{ w \in \left\{ a, b \right\}^{*} \mid  \left| w \right|_{b} \text{кратно трем} \right\} $

$ \Sigma_{B} = \left\{ a, b \right\}$
$ Q_{B} = \left\{ 1, 2, 3 \right\} $
$ s_{B} = \left\{ 1 \right\} $
$ T_{B} = \left\{ 1 \right\}$

\digraph[scale=0.5]{esm23B3}{
	rankdir=LR; 
	node [shape=none]; start;
	node [shape=circle]; 2 3;
	node [shape=doublecircle]; 1;
	start -> 1;
	1 -> 1 [label = "a"];
	1 -> 2 [label = "b"];
	2 -> 2 [label = "a"];
	2 -> 3 [label = "b"];
	3 -> 3 [label = "a"];
	3 -> 1 [label = "b"];
}

Тогда: $L_{3} = A_{3} \times B_{3}$

$ \Sigma_{3} = \left\{ a, b \right\}$

$ Q_{3} = \left\{ 11, 12, 13, 21, 22, 23 \right\}$

$ s_{3} = \left\{ 11 \right\}$

$ T_{3} = \left\{ 11 \right\}$


\begin{table}[!htbp]
\rowcolors{2}{gray!10}{gray!40}
\centering
\begin{tabular}{|l|l|l|l|}
\hline
$A$ & $B$ & $a$ & $b$ \\ \hline
1   & 1   & 21  & 12  \\ \hline
1   & 2   & 22  & 13  \\ \hline
1   & 3   & 23  & 11  \\ \hline
2   & 1   & 11  & 22  \\ \hline
2   & 2   & 12  & 23  \\ \hline
2   & 3   & 13  & 21  \\ \hline
\end{tabular}
\caption{Таблица состояний (Задание 2.3)}
\end{table}


\digraph[scale=0.5]{esm23}{
	rankdir=LR; 
	node [shape=none]; start;
	node [shape=circle]; 12 13 21 22 23;
	node [shape=doublecircle]; 11;
	start -> 11;
	11 -> 12 [label = "b"];
	11 -> 21 [label = "a"];
	12 -> 13 [label = "b"];
	12 -> 22 [label = "a"];
	13 -> 11 [label = "b"];
	13 -> 23 [label = "a"];

	21 -> 22 [label = "b"];
	21 -> 11 [label = "a"];
	22 -> 23 [label = "b"];
	22 -> 12 [label = "a"];
	23 -> 21 [label = "b"];
	23 -> 13 [label = "a"];
}

\subsection{Язык 4.}
$$ L_{4} = \bar{L_{3}}$$

Данный язык будет распознаяаться автоматом 
$ \bar{L_{3}} = \left\{ \Sigma_{3}, Q_{3}, s_{3}, Q_{3}\backslash T_{3}, \delta_{3} \right\} $

$ T_{4} = Q_{3} \backslash T_{3} = \left\{ 12, 13, 21, 22, 23 \right\} $


\digraph[scale=0.5]{esm24}{
	rankdir=LR; 
	node [shape=none]; start;
	node [shape=doublecircle]; 12 13 21 22 23;
	node [shape=circle]; 11;
	start -> 11;
	11 -> 12 [label = "b"];
	11 -> 21 [label = "a"];
	12 -> 13 [label = "b"];
	12 -> 22 [label = "a"];
	13 -> 11 [label = "b"];
	13 -> 23 [label = "a"];

	21 -> 22 [label = "b"];
	21 -> 11 [label = "a"];
	22 -> 23 [label = "b"];
	22 -> 12 [label = "a"];
	23 -> 21 [label = "b"];
	23 -> 13 [label = "a"];
}


\subsection{Язык 5.}
$$ L_{5} = L_{2} \backslash L_{3}$$


$ L_{5} = L_{2} \backslash L_{3} = L_{2} \cap \bar{L_{3}} = L_{2} \times \bar{L_{3}} $

Заметим, что автомат для $ L_{2} $ можно упростить:

\digraph[scale=0.5]{esm25L2}{
	rankdir=LR; 
	node [shape=none]; start;
	node [shape=circle]; 1 2 3;
	node [shape=doublecircle]; 4;
	start -> 1;
	1 -> 2 [label = "a,b"];
	2 -> 3 [label = "a,b"];
	3 -> 4 [label = "a,b"];
	4 -> 3 [label = "a,b"];
}

Для автомата $ \bar{L_{3}} $ введем новую нумерацию состояний:

\digraph[scale=0.5]{esm25nL3}{
	rankdir=LR; 
	node [shape=none]; start;
	node [shape=doublecircle]; 2 3 4 5 6;
	node [shape=circle]; 1;
	start -> 1;
	1 -> 2 [label = "b"];
	1 -> 4 [label = "a"];
	2 -> 3 [label = "b"];
	2 -> 5 [label = "a"];
	3 -> 1 [label = "b"];
	3 -> 6 [label = "a"];

	4 -> 5 [label = "b"];
	4 -> 1 [label = "a"];
	5 -> 6 [label = "b"];
	5 -> 2 [label = "a"];
	6 -> 4 [label = "b"];
	6 -> 3 [label = "a"];
}


\begin{table}[!htbp]
\rowcolors{2}{gray!10}{gray!40}
\centering
\begin{tabular}{|l|l|l|l|}
\hline
$L_{2}$ & $\bar{L_{3}}$ & $a$ & $b$ \\ \hline
1       & 1             & 24  & 22  \\ \hline
1       & 2             & 25  & 23  \\ \hline
1       & 3             & 26  & 21  \\ \hline
1       & 4             & 21  & 25  \\ \hline
1       & 5             & 22  & 26  \\ \hline
1       & 6             & 23  & 24  \\ \hline
2       & 1             & 34  & 32  \\ \hline
2       & 2             & 35  & 33  \\ \hline
2       & 3             & 36  & 31  \\ \hline
2       & 4             & 31  & 35  \\ \hline
2       & 5             & 32  & 36  \\ \hline
2       & 6             & 33  & 34  \\ \hline
3       & 1             & 44  & 42  \\ \hline
3       & 2             & 45  & 43  \\ \hline
3       & 3             & 46  & 41  \\ \hline
3       & 4             & 41  & 45  \\ \hline
3       & 5             & 42  & 46  \\ \hline
3       & 6             & 43  & 44  \\ \hline
4       & 1             & 34  & 32  \\ \hline
4       & 2             & 35  & 33  \\ \hline
4       & 3             & 36  & 31  \\ \hline
4       & 4             & 31  & 35  \\ \hline
4       & 5             & 32  & 36  \\ \hline
4       & 6             & 33  & 34  \\ \hline
\end{tabular}
\caption{Таблица состояний (Задание 2.3)}
\end{table}



Получился такой автомат:

\digraph[scale=0.3]{esm25}{
	rankdir=LR; 
	node [shape=none]; start;
	node [shape=doublecircle]; 42 43 45 46;
	node [shape=circle]; 11 12 13 14 15 16 21 22 23 24 25 26 31 32 33 34 35 36 41 44;
	start -> 11;
	11 -> 24 [label="a"]
	12 -> 25 [label="a"]
	13 -> 26 [label="a"]
	14 -> 21 [label="a"]
	15 -> 22 [label="a"]
	16 -> 23 [label="a"]
	21 -> 34 [label="a"]
	22 -> 35 [label="a"]
	23 -> 36 [label="a"]
	24 -> 31 [label="a"]
	25 -> 32 [label="a"]
	26 -> 33 [label="a"]
	31 -> 44 [label="a"]
	32 -> 45 [label="a"]
	33 -> 46 [label="a"]
	34 -> 41 [label="a"]
	35 -> 42 [label="a"]
	36 -> 43 [label="a"]
	41 -> 34 [label="a"]
	42 -> 35 [label="a"]
	43 -> 36 [label="a"]
	44 -> 31 [label="a"]
	45 -> 32 [label="a"]
	46 -> 33 [label="a"]



	11 -> 22 [label="b"]
	12 -> 23 [label="b"]
	13 -> 21 [label="b"]
	14 -> 25 [label="b"]
	15 -> 26 [label="b"]
	16 -> 24 [label="b"]
	21 -> 32 [label="b"]
	22 -> 33 [label="b"]
	23 -> 31 [label="b"]
	24 -> 35 [label="b"]
	25 -> 36 [label="b"]
	26 -> 34 [label="b"]
	31 -> 42 [label="b"]
	32 -> 43 [label="b"]
	33 -> 41 [label="b"]
	34 -> 45 [label="b"]
	35 -> 46 [label="b"]
	36 -> 44 [label="b"]
	41 -> 32 [label="b"]
	42 -> 33 [label="b"]
	43 -> 31 [label="b"]
	44 -> 35 [label="b"]
	45 -> 36 [label="b"]
	46 -> 34 [label="b"]

}


Упростим его:

\digraph[scale=0.3]{esm25v2}{
	rankdir=LR; 
	node [shape=none]; start;
	node [shape=doublecircle]; 42 43 45 46;
	node [shape=circle]; 11 21 22 24 31 32 33 34 35 36 41 44;
	start -> 11;
	11 -> 24 [label="a"]
	21 -> 34 [label="a"]
	22 -> 35 [label="a"]
	24 -> 31 [label="a"]
	31 -> 44 [label="a"]
	32 -> 45 [label="a"]
	33 -> 46 [label="a"]
	34 -> 41 [label="a"]
	35 -> 42 [label="a"]
	36 -> 43 [label="a"]
	41 -> 34 [label="a"]
	42 -> 35 [label="a"]
	43 -> 36 [label="a"]
	44 -> 31 [label="a"]
	45 -> 32 [label="a"]
	46 -> 33 [label="a"]



	11 -> 22 [label="b"]
	21 -> 32 [label="b"]
	22 -> 33 [label="b"]
	24 -> 35 [label="b"]
	31 -> 42 [label="b"]
	32 -> 43 [label="b"]
	33 -> 41 [label="b"]
	34 -> 45 [label="b"]
	35 -> 46 [label="b"]
	36 -> 44 [label="b"]
	41 -> 32 [label="b"]
	42 -> 33 [label="b"]
	43 -> 31 [label="b"]
	44 -> 35 [label="b"]
	45 -> 36 [label="b"]
	46 -> 34 [label="b"]

}


\section{Задание №3. Построить минимальный ДКА по регулярному выражению.}


\subsection{Регулярное выражение 1.}
$$ (ab + aba)^{*}a $$


\digraph[scale=0.5]{esm31v1}{
	rankdir=LR; 
	node [shape=none]; start;
	node [shape=doublecircle]; 6;
	node [shape=circle]; 1 2 3 4 5;
	start -> 1;
	1 -> 2 [label = "a"];
	1 -> 3 [label = "a"];
	1 -> 5 [label = "lmb"];
	5 -> 1 [label = "lmb"];

	3 -> 4 [label = "b"];
	2 -> 5 [label = "b"];
	4 -> 5 [label = "a"];
	5 -> 6 [label = "a"];
}

Детерминируем этот автомат:

\digraph[scale=0.5]{esm31v2}{
	rankdir=LR; 
	node [shape=none]; start;
	node [shape=doublecircle]; 236 23516;
	node [shape=circle]; 15 145;
	start -> 15;
	15 -> 236 [label = "a"];
	236 -> 145 [label = "b"];
	145 -> 23516 [label = "a"];
	23516 -> 145 [label = "b"];
	23516 -> 236 [label = "a"];
}


\subsection{Регулярное выражение 2.}
$$ a(a (ab)^{*} b)^{*} (ab)^{*} $$


\digraph[scale=0.5]{esm32v1}{
	rankdir=LR; 
	node [shape=none]; start;
	node [shape=doublecircle]; 8;
	node [shape=circle]; 1 2 3 4 5 6 7;
	start -> 1;
	1 -> 2 [label = "a"];
	2 -> 3 [label = "a"];
	2 -> 6 [label = "lmb"];
	6 -> 2 [label = "lmb"];
	3 -> 4 [label = "a"];
	3 -> 5 [label = "lmb"];
	5 -> 3 [label = "lmb"];
	4 -> 5 [label = "b"];
	5 -> 6 [label = "b"];
	6 -> 7 [label = "a"];
	6 -> 8 [label = "lmb"];
	8 -> 6 [label = "lmb"];
	7 -> 8 [label = "b"];
}

Детерминируем этот автомат:

\digraph[scale=0.5]{esm32v2}{
	rankdir=LR; 
	node [shape=none]; start;
	node [shape=doublecircle]; 268;
	node [shape=circle]; 1 357 4 35;
	start -> 1;
	1 -> 268 [label = "a"];
	268 -> 357 [label = "a"];
	357 -> 4 [label = "a"];
	357 -> 268 [label = "b"];
	4 -> 35 [label = "b"];
	35 -> 4 [label = "a"];
	35 -> 268 [label = "b"];
}


\subsection{Регулярное выражение 3.}
$$ (a + (a + b)(a + b)b)^{*} $$


\digraph[scale=0.5]{esm33v1}{
	rankdir=LR; 
	node [shape=none]; start;
	node [shape=doublecircle]; 4;
	node [shape=circle]; 1 2 3;
	start -> 1;
	1 -> 2 [label = "a"];
	1 -> 2 [label = "b"];
	1 -> 4 [label = "lmb"];
	1 -> 4 [label = "lmb"];
	1 -> 4 [label = "a"];
	2 -> 3 [label = "a"];
	2 -> 3 [label = "b"];
	3 -> 4 [label = "b"];
}

Детерминируем этот автомат:

\digraph[scale=0.5]{esm33v2}{
	rankdir=LR; 
	node [shape=none]; start;
	node [shape=doublecircle]; 14 412 341;
	node [shape=circle]; 23 2 3;
	start -> 14;
	14 -> 412 [label = "a"];
	14 -> 2 [label = "b"];
	412 -> 412 [label = "a"]
	412 -> 23 [label = "b"]
	23 -> 3 [label = "a"]
	23 -> 341 [label = "b"]
	341 -> 14 [label = "a"]
	341 -> 412 [label = "b"]
	3 -> 14 [label = "b"]
	2 -> 3 [label = "a"]
	2 -> 3 [label = "b"]
}

\subsection{Регулярное выражение 4.}
$$ (b + c) ((ab)^{*}c + (ba)^{*})^{*} $$


\digraph[scale=0.5]{esm34v1}{
	rankdir=LR; 
	node [shape=none]; start;
	node [shape=doublecircle]; 10;
	node [shape=circle]; 1 2 3 4 5 6 7 8 9;
	start -> 1;
	1 -> 2 [label = "b"];
	1 -> 2 [label = "c"];
	2 -> 3 [label = "lmb"];
	2 -> 7 [label = "lmb"];
	2 -> 10 [label = "lmb"];
	3 -> 4 [label = "a"]
	3 -> 5 [label = "lmb"]
	4 -> 5 [label = "b"]
	5 -> 6 [label = "c"]
	5 -> 3 [label = "lmb"]
	6 -> 10 [label = "lmb"]
	7 -> 8 [label = "b"]
	7 -> 9 [label = "lmb"]
	8 -> 9 [label = "a"]
	9 -> 7 [label = "lmb"]
	9 -> 10 [label = "lmb"]
	10 -> 2 [label = "lmb"];
}

Детерминируем этот автомат:

\digraph[scale=0.5]{esm34v2}{
	rankdir=LR; 
	node [shape=none]; start;
	node [shape=doublecircle]; 2357910 62357910;
	node [shape=circle]; 1 4 53 8;
	start -> 1;
	1 -> 2357910 [label = "b"];
	1 -> 2357910 [label = "c"];
	2357910 -> 4 [label = "a"];
	2357910 -> 8 [label = "b"];
	2357910 -> 62357910 [label = "c"];
	62357910 -> 4 [label = "a"];
	62357910 -> 8 [label = "b"];
	62357910 -> 62357910 [label = "c"];
	4 -> 53 [label = "b"];
	53 -> 4 [label = "a"];
	53 -> 62357910 [label = "c"];
	8 -> 2357910 [label = "a"];
}

\subsection{Регулярное выражение 5.}
$$ (a+b)^{+}(aa+bb+abab+baba)(a+b)^{+} $$


\digraph[scale=0.5]{esm35v1}{
	rankdir=LR; 
	node [shape=none]; start;
	node [shape=doublecircle]; 12;
	node [shape=circle]; 1 2 3 4 5 6 7 8 9 10 11;
	start -> 1;
	1 -> 2 [label = "a"];
	1 -> 2 [label = "b"];
	2 -> 3 [label = "a"];
	2 -> 3 [label = "b"];
	2 -> 3 [label = "lmb"];
	3 -> 2 [label = "lmb"];
	3 -> 4 [label = "a"];
	3 -> 7 [label = "b"];
	4 -> 10 [label = "a"];
	4 -> 5 [label = "b"];
	5 -> 6 [label = "a"];
	6 -> 10 [label = "b"];
	7 -> 8 [label = "a"];
	7 -> 10 [label = "b"];
	8 -> 9 [label = "b"];
	9 -> 10 [label = "a"];
	10 -> 11 [label = "a"];
	10 -> 11 [label = "b"];
	11 -> 12 [label = "a"];
	11 -> 12 [label = "b"];
	11 -> 12 [label = "lmb"];
	12 -> 11 [label = "lmb"];
}


\digraph[scale=0.2]{esm35v2}{
	rankdir=LR; 
	node [shape=none]; start;
	node [shape=doublecircle]; 234101112 234861112 23759101112 23751112 237101112 23486101112 23481112 237591112;
	node [shape=circle]; 1 23 234 237 23410 2375 2348 23710 23486 23759 2375910 2348610;
	start -> 1;
	1 -> 23 [label = "a"];
	1 -> 23 [label = "b"];
	23 -> 234 [label = "a"];
	23 -> 237 [label = "b"];
	234 -> 23410 [label = "a"]
	234 -> 2375 [label = "b"]
	237 -> 2348 [label = "a"]
	237 -> 23710 [label = "b"]
	23410 -> 234101112 [label = "a"]
	23410 -> 23751112 [label = "b"]
	2375 -> 23486 [label = "a"]
	2375 -> 23710 [label = "b"]
	2348 -> 23410 [label = "a"]
	2348 -> 23759 [label = "b"]
	23710 -> 23481112 [label = "a"]
	23710 -> 237101112 [label = "b"]
	234101112 -> 234101112 [label = "a"]
	234101112 -> 23751112 [label = "b"]
	23751112 -> 234861112 [label = "a"]
	23751112 -> 237101112 [label = "b"]
	23486 -> 23410 [label = "a"]
	23486 -> 2375910 [label = "b"]
	23759 -> 2348610 [label = "a"]
	23759 -> 23710 [label = "b"]
	23481112 -> 234101112 [label = "a"]
	23481112 -> 237591112 [label = "b"]
	237101112 -> 23481112 [label = "a"]
	237101112 -> 237101112 [label = "b"]
	234861112 -> 234101112 [label = "a"]
	234861112 -> 23759101112 [label = "b"]
	2375910 -> 23486101112 [label = "a"]
	2375910 -> 237101112 [label = "b"]
	2348610 -> 234101112 [label = "a"]
	2348610 -> 23759101112 [label = "b"]
	237591112 -> 23486101112 [label = "a"]
	237591112 -> 237101112 [label = "b"]
	23759101112 -> 23486101112 [label = "a"]
	23759101112 -> 237101112 [label = "b"]
	23486101112 -> 234101112 [label = "a"]
	23486101112 -> 23759101112 [label = "b"]
}

\section{Задание №4. Определить является ли язык регулярным или нет.}


\subsection{Язык 1.}
$$ L = \left\{ (aab)^{n}b(aba)^{m} \mid n \geq 0, m \geq 0 \right\} $$


\digraph[scale=0.5]{esm41}{
	rankdir=LR; 
	node [shape=none]; start;
	node [shape=doublecircle]; 8;
	node [shape=circle]; 1 2 3 4 5 6 7;
	start -> 1;
	1 -> 2 [label = "a"];
	1 -> 4 [label = "lmb"];
	2 -> 3 [label = "a"];
	3 -> 4 [label = "b"];
	4 -> 5 [label = "b"];
	4 -> 1 [label = "lmb"];
	5 -> 6 [label = "a"];
	5 -> 8 [label = "lmb"];
	6 -> 7 [label = "b"];
	7 -> 8 [label = "a"];
	8 -> 5 [label = "lmb"];
}



\subsection{Язык 2.}
$$ L = \left\{ uaav \mid u \in \left\{ a, b \right\}^{*}, v \in \left\{ a, b \right\}^{*}, \left| u \right|_{b} \geq \left| v \right|_{a} \right\} $$


Пусть $ \bar{L} = \left\{ uaav \mid u \in \left\{ a, b \right\}^{*}, v \in \left\{ a, b \right\}^{*}, \left| u \right|_{b} < \left| v \right|_{a} \right\} $

Фиксируем $ n $

Пусть $ w = b^{n} aa a^{n+1}; \quad \left| w \geq n \right| $ (Взяли такое разбиение, так как количество букв $ a $ в $ u $ и количество букв $ b $ в $ v $ не важны)

$ x = b^{l} $ 

$ y = b^{p} $

$ z = b^{n-l-p}aaa^{n+1} $

$ xy^{k}z = b^{n+p(k-1)}aaa^{n+1} $ (при $ k \geq 2 \quad w \notin \bar{L} $) $ \Rightarrow  \bar{L} $ - нерегулярный язык, следовательно, $ L $ также не является регулярным.



\subsection{Язык 3.}
$$ L = \left\{ a^{m}w \mid w \in \left\{ a, b \right\}^{*}, 1 \leq \left| w \right|_{b} \leq m \right\} $$

Пусть $ \bar{L} = \left\{ a^{m}w \mid w \in \left\{ a, b \right\}^{*},  \left| w \right|_{b} > m \right\} $

$ w_{1} = a^{m}b^{n} $, $ \left| w_{1} \geq n \right| $ - выполнено.

$ x = a^{p} $

$ y = a^{l} $

$ z = a^{m-l-p}b^{n} $

$ p + l < n $ - выполнено, так как, по условию языка $ \bar{L}  \quad m < n$

$ w_{1}' = xy^{k}z = a^{m+l(k-1)}b^{n} \Rightarrow$ при $ k \geq 2 \quad w_{1}' \notin \bar{L} \Rightarrow \bar{L}$ нерегулярный язык $ \Rightarrow \quad L$ - нерегулярный язык.



\subsection{Язык 4.}
$$ L = \left\{ a^{k}b^{n}a^{n} \mid k = n \vee m > 0 \right\}$$

1) $ k = n \quad \Rightarrow L_{1} = \left\{ a^{n}b^{m}a^{n} \right\} $
\bigskip

$ w_{1} = a^{n}b^{m}a^{n} \quad \left| w_{1} \right| \geq n$

$ x = a^{l} $

$ y = a^{p} $

$ z = a^{n-l-p}b^{m}a^{n} $

\bigskip
$ l+p \leq n $

$ p \geq 1 $

\bigskip
$ w_{1}' = xy^{t}z = a^{n+p(t-1)}b^{m}a^{n} \quad \Rightarrow \quad w_{1}' \notin L_{1} $ при $ t \geq 2 $

\bigskip
2) $ k = m \quad \Rightarrow L_{2} = \left\{ a^{m}b^{m}a^{n} \right\} $
\bigskip

$ w_{2} = a^{m}b^{m}a^{n} \quad \left| w_{2} \right| \geq m$

$ x = a^{l} $

$ y = a^{p} $

$ z = a^{m-l-p}b^{m}a^{n} $
\bigskip

$ l+p \leq m $

$ p \geq 1 $
\bigskip

$ w_{2}' = xy^{t}z = a^{m+p(t-1)}b^{m}a^{n} \quad \Rightarrow \quad w_{2}' \notin L_{2} $ при $ t \geq 2 $
\bigskip

Из 1), 2) $ \Rightarrow $ язык $ L $ - нерегулярный.
\subsection{Язык 5.}
$$ L = \left\{ ucv \mid u \in \left\{ a, b \right\}^{*}, v \in \left\{ a, b \right\}^{*}, u \neq v^{R}\right\} $$

Пусть $ \bar{L} = \left\{ ucv \mid u \in \left\{ a, b \right\}^{*}, v \in \left\{ a, b \right\}^{*}, u = v^{R}\right\}  $

Фиксируем $ n $: $ \left| u \right| \leq n$

При разделении $ u $ на $ x $ и $ y $ и накачке перестанет выполнятся равенство $ u = v^{R} $, поэтому язык $ \bar{L} $ нерегулярный $ \Rightarrow $ язык $ L $ также не является регулярным. 

\section{Задание №5. Реализовать алгоритмы.}
В рамках своего выполенения программа должна генерировать текстовый документ с картинками, показывающий процесс построения автомата (к примеру, Markdown с графиками на Graphviz).

\begin{enumerate}
	\item Построение ДКА по НКА с $\lambda$-переходами.
	\item Прямое произведение языков, с возможностью построить пересечение, объединение и разность.
\end{enumerate}

\end{document}
